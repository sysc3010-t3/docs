\documentclass[letterpaper,12pt]{report}

\usepackage{graphicx}
\usepackage{hyperref}
\usepackage[top=1in, bottom=1in, left=1.25in, right=1in]{geometry}

\hypersetup{
    colorlinks, %set true if you want colored links
    linkcolor=black,  %choose some color if you want links to stand out
}

\begin{document}
	\title{SYSC3010 T3 Project Proposal}
	\author{RC Camera Car}
	\date{}
	\maketitle

	\begin{abstract}
		The students developing this project are Alec D'Alessandro, Thao-Tran
		Le-Phuong, Honor Lopes, and Igor Veselinovic. This document will
		outline the motivations, goals, and the proposed solution for the
		project, as well as the motivations for pursuing this project. The
		proposal will also provide the projected plan and milestones for the
		project.
	\end{abstract}

	\tableofcontents

	\pagebreak

	\section*{Motivation}
	\markright{}
	\addcontentsline{toc}{section}{Motivation}
	The purpose of this project is to create a remote-controlled vehicle in
	order to explore uninhabited or unreachable locations. There are many
	locations inaccessible or unsafe for manned vehicles such as warzones,
	cave systems, and biohazard spaces. These are locations where it would be
	unsafe to send human drives and would be preferable to have the vehicle
	operated remotely. This project would allow us to explore potentially
	dangerous regions without risk to human safety.

	The planned vehicle will have a camera attached in order for it to obtain
	video of the vehicles surroundings. This allows the user to obtain data on
	inaccessible locations. This video feed would also allow the vehicle to be
	used remotely. The automatic light sensor system would allow the vehicle to
	be operated at night and in low light conditions. This project has science,
	military and recreational applications.

	\section*{Goals/Scope}
	\markright{}
	\addcontentsline{toc}{section}{Goals/Scope}
	The first goal for this project is for the car to be controlled via
	smart-phone through our designed android application. Once the user has
	opened the application there will be a live feed video source that is being
	transmitted from the car. This will allow the user to view from the
	perspective of sitting on top of the car. The car could then be outfitted
	with additional auxiliaries needed for specific jobs. Examples of this
	include a shovel to take samples of soil or an arm to pick up and move
	target objects.

	One limitation to the project development is how much bandwidth can be used
	for the live video feed stream from the car to the android application. A
	proposed solution to this would be to decrease video resolution for
	bandwidth reducing purposes, which would lead to mal visibility. A
	self-imposed constraint is on the complexity of the development of the car.
	This project will be focused on the vehicle controls and video feed. If the
	longevity of the project permitted more components would be added to ensure
	the car would give benefit to various jobs and a wide variety of terrains.

	\section*{Objectives}
	\markright{}
	\addcontentsline{toc}{section}{Objectives}
	\begin{itemize}
		\item Remote controlled vehicle
		\item Video feed from camera on vehicle
		\item Automatic headlights
		\item Brightness sensors
		\item Video feed displayed on android app
		\item Vehicle controls on android app
		\item Accounts and user authentication
	\end{itemize}

	\section*{Proposed Solution}
	\markright{}
	\addcontentsline{toc}{section}{Proposed Solution}
	Users will use the Android app to register their car to themselves and
	connect it to a Wi-Fi network. After the car is connected to Wi-Fi, users
	can control the car’s movements via the app as well as stream a video feed
	from and control the mounted camera. The video feed is to allow the user to
	control the car when it is not in sight. The app will also allow users to
	control the headlights of the car in case of dimly lit settings.

	There will be a Raspberry Pi that acts as the centralized server. The
	server will have a database that will store the users, cars, and the
	pairings of users to cars. It will handle all of the commands coming from
	users and route them to their associated cars. Video feed from each car
	will be sent to the server to be sent to the proper user and can optionally
	be stored on a hard drive at the user’s request.

	The body of the car will contain an Arduino and a Raspberry Pi. The Arduino
	will control the movement of the car (steering, acceleration) and handle
	the headlights of the car (LDR sensor and LEDs). The LDR sensor will be
	used to detect when an area is dimly lit and the Arduino will automatically
	turn on the headlights (LEDs). The car movement will require motors to be
	connected to the Arduino and possibly a custom 3D printed chassis to hold
	all of the components. The car will use continuous tracks for propulsion as
	they will allow the car to traverse more types of terrain. The Raspberry Pi
	will be connected to a Wi-Fi network to receive user commands and relay
	them to the Arduino via a wired connection. The Raspberry Pi will also be
	connected to the camera and will stream the video feed to the centralized
	server.

	The system design is shown in Figure \ref{fig:uml}. The UML diagram outlines the
	components necessary for implementation and the relationships between them.

	\begin{figure}[h]
    	\centering
		\includegraphics[width=\linewidth]{Proposal_UML_Diagram.png}
    	\caption{UML Diagram for RC Camera Car System}
    	\label{fig:uml}
	\end{figure}

	\section*{Project Plan}
	\markright{}
	\addcontentsline{toc}{section}{Project Plan}
	Our remote-controlled car system has three main components that can be
	developed mostly in parallel. The car itself and the basic I/O components
	that provide the core functionality (i.e, the motors, light sensors, and
	headlights) make up one grouping of work. All handling of video, including
	interfacing with the camera, streaming, and long-term storage, is another
	main thread of work. The last central component to this system is the
	Android application which must have a GUI and needs to be tested
	differently since it is user-facing. Testing is a vital component of each
	stream of work and will be done throughout the development process of all
	software.

	There are obvious interdependencies between section of work such as the
	Android application needing to receive video from the Raspberry Pi server
	and the Android application sending movement commands to the car. Even
	after testing components independently, new bugs and issues might arise
	once separate components are integrated together. Thus, beyond the unit
	testing of individual components, integration testing will need to be done
	at every stage of the system’s development.

	\subsection*{Milestones}
	\markright{}
	\addcontentsline{toc}{subsection}{Milestones}
	\underline{Legend}:
	\begin{itemize}
		\item[] \textbf{CAR}: Remote-controlled car
		\item[] \textbf{VID}: Video streaming and storage
		\item[] \textbf{AND}: Android application
		\item[] \textbf{TST}: Testing
	\end{itemize}
	\noindent\rule{\textwidth}{0.5pt}
	\begin{itemize}
		\item \textbf{[CAR]} Acquire/build all components for the system (camera, tracks/wheels, chassis, motor, etc.)
		\item \textbf{[CAR]} Develop a codebase for controlling the car’s wheels/tracks using motors to allow for forward/backward movement and turning
		\item \textbf{[CAR, TST]} Develop a comprehensive unit test suite of the car movement control code
		\item \textbf{[CAR, TST]} Manually test and fine-tune the car movement control code
	\end{itemize}
	\noindent\rule{\textwidth}{0.5pt}
	\begin{itemize}
		\item \textbf{[CAR]} Set up periodic checks of environment light levels using the light sensor and adjust the brightness of the headlights accordingly
		\item \textbf{[CAR, TST]} Develop a comprehensive unit test suite of the light control code
	\end{itemize}
	\noindent\rule{\textwidth}{0.5pt}
	\begin{itemize}
		\item \textbf{[CAR]} Send movement commands from centralized Raspberry Pi server to the car’s Raspberry Pi, relay them to the Arduino, and power the corresponding motor(s)
		\item \textbf{[CAR, TST]} Develop a comprehensive unit test suite of the networking code
	\end{itemize}
	\noindent\rule{\textwidth}{0.5pt}
	\begin{itemize}
		\item \textbf{[VID]} Send video data from the car’s Raspberry Pi to the centralized Raspberry Pi and relay the stream to clients
		\item \textbf{[VID]} Store video clips on an external hard drive and track the files using an SQL database
		\item \textbf{[VID, TST]} Develop a comprehensive functional test suite of the video streaming and storage code
	\end{itemize}
	\noindent\rule{\textwidth}{0.5pt}
	\begin{itemize}
		\item \textbf{[AND, CAR]} Send commands to the centralized Raspberry Pi server based on inputs to the Android application GUI that will then be relayed to the car
		\item \textbf{[AND, VID]} Receive and play video streams and recorded clips from the centralized Raspberry Pi server in the Android application GUI
		\item \textbf{[AND, TST]} Develop a sanity test suite of the GUI of the Android application
		\item \textbf{[AND, TST]} Develop a comprehensive unit test suite of the Android application code
	\end{itemize}
	\noindent\rule{\textwidth}{0.5pt}
\end{document}
